%%%%%%% The majority of notes and code is from the original AU template. Adjustments/Additions are indicated by SH (Simon Heuberger)
%Last modified: 2018 Apr 30
% SH modifications for Markdown made in February 2019

\documentclass[12pt,econ]{sources/authesis}
% IMPORTANT NOTES:
% 1) You MUST run LaTeX THREE times after runing BibTeX.
% 2) Make sure that you have the LATEST version of the AUTHESIS class files
%    before you hand in the final version of your thesis or dissertation.
% 3) SPA students can uncomment the next two lines:
%\renewcommand{\schoolorcollege}{School of Public Affairs}
%\renewcommand{\deansigline}{\parbox{2.1in}{Dean of the School}}

% \usepackage commands should go before the \begin{document}
%required packages
\usepackage{natbib}  % natbib citation style

%%% SH begin
% Removed \usepackage{url} and replaced it with the more functional {hyperref}
\usepackage[hidelinks]{hyperref}
%%% SH end

%optional packages

%CTABLE
%\usepackage{ctable} %You should learn about ctable!!

%ULEM
% IF you use the 'cas' option instead of the preferred 'econ' option,
% you MAY uncomment the following line.  (NOT recommended.)
%\usepackage{ulem}


%%% SH begin
% Removed the longtable environment and the longtable in chapter 01 of the template because it is very bad practice
% The university itself strongly advises against using longtables in the template and only provides it as a last resort

% The following packages are for the LaTeX generated graph
\usepackage{lscape}
\usepackage{float} % to float stargazer tables with argument [t] specified in the AU thesis class

% FIGURES
%

% The following is repeated from the .cls file to make it appear in Markdown. The commands set the captions and also make the lof and lot double-spaced (unless it extends to more than one line, in which case it's single-spaced)
\makeatletter
\def\caption{\refstepcounter\@captype \@dblarg{\@caption\@captype}}
\long\def\@caption#1[#2]#3{%
  \par \par
  \addcontentsline{\csname ext@#1\endcsname}{#1}%
    {\protect\numberline{\csname the#1\endcsname.}{\ignorespaces #2}}%
  \addtocontents{\csname ext@#1\endcsname}{\protect\addvspace{10\p@}}%
  \begingroup
    \@parboxrestore
    \if@minipage
      \@setminipage
    \fi
    \normalsize
    \@makecaption{\csname fnum@#1\endcsname}{\ignorespaces #3}\par
  \endgroup}
\makeatother


\title{MY THESIS OR DISSERTATION TITLE IN CAPITALS\newline (WITH 48 CHARACTERS
OR FEWER PER LINE)\newline AS AN INVERTED PYRAMID}
\author{Will U. Finnish}
\degreeyear{2525}
\degree{Doctor of Philosophy}
\chair{Professor Ivory Tower}
\secondreader{Professor Ivory Tower}
\thirdreader{Professor Mih Sing Cite}
\fourthreader{Professor Ver Tigo}
\degreefield{Introspective Empiricism}
%%% SH end
\begin{document}

% Declarations for Front Matter
%
% IMPORTANT: IF A TITLE OR SUBTITLE EXCEEDS MORE THAN ONE LINE,
%            THERE SHOULD ONLY BE  48 CHARACTERS PER LINE.
%
% The dissertation title must be capitalized for AU-CAS

%%% SH begin
% \title{}, \author{}, \degree{}, \fourthreader etc. are all specified in the .Rmd. Any changes to the number of readers will also need to be made in "authesis.cls", like in the original template
%%% SH end

% The following command makes the title page, it is duplicated to produce
% two copies of cover page, as required by AU-CAS.

\maketitle

\maketitle

% The following command makes the copyright page

\copyrightpage
\begin{frontmatter}
% The Guide requires one numbered abstract page.
% The 'abstractn' macro generates a numbered abstract page.
% Do NOT use the 'abstract' macro, which generates an unnumbered, abstract page.
% (NOTE: The Guide no longer requires two abstract pages, one numbered, one without numbers.)

%%% SH begin
\abstractn{This is an abstractn. It has a page number. (AU no longer wants you to
provide an unnumbered abstract page.) The Guide says an abstract should
not exceed 350 words. (If you do exceed this, then under the CAS option,
the first page is numbered bottom center, and the rest are numbered top
right.) This is an abstractn. This is an abstractn. This is an
abstractn. This is an abstractn. This is an abstractn. This is an
abstractn. This is an abstractn. This is an abstractn. This is an
abstractn. This is an abstractn. This is an abstractn. This is an
abstractn. This is an abstractn. This is an abstractn. This is an
abstractn. This is an abstractn. This is an abstractn. This is an
abstractn. This is an abstractn. This is an abstractn. This is an
abstractn. This is an abstractn. This is an abstractn. This is an
abstractn. This is an abstractn.}
% The content of the abstract is specified in the .Rmd
%%% SH end


% Acknowledgements are optional. If you do not wish to include them,
% simply do not include this command in your source.

%%% SH begin
\acknowledgements{I want to ``thank'' my committee, without whose ridiculous demands I
would have graduated so very much faster.}
% The content of the acknowledgements is specified in the .Rmd
%%% SH end

\listoftables

\listoffigures

%%% SH begin
% lot and lof moved before the toc because \setcounter{page}{1} in the frontmatter environment for some reason messes up the toc page numbering when they come after the toc. This is cleared and acceptable with AU admin
%%% SH end

\tableofcontents

\end{frontmatter}


%%% SH begin
\chapter{INTRODUCTION}\label{introduction}

Every dissertation should have an introduction. The introduction should
introduce the concepts, background, and goals of the dissertation.

\section{\texorpdfstring{My First Section Title \newline With 48 or
Fewer Characters Per
Line}{My First Section Title With 48 or Fewer Characters Per Line}}\label{my-first-section-title-with-48-or-fewer-characters-per-line}

This is the first sentence of the first paragraph of the first section.
Cool, eh? This paragraph includes inline higher mathematics: \(y=f(x)\).
This paragraph also includes a display equation:
\begin{equation}
h = f \circ g
\end{equation}
Furthermore, this section includes a floating table.

Another paragraph. Another sentence. Another sentence. Another sentence.
Another sentence. Another sentence. Another sentence. Another sentence.
Another sentence. Another sentence. Another sentence. Another sentence.
Another sentence. Another sentence. Another sentence. \footnote{
Here is a footnote.}
\begin{table}[ht]\centering
\caption{A normalsize table.
Captions for tables must go above the table.
Table captions must be single-spaced,
and the code indicates that they should be formatted as such.}
\begin{tabular}{lr}\hline\hline
Title & Author \\ \hline
TANSTAAFL & Milton Friedman \\
Oh Yes There Is & John Maynard Keynes \\ \hline
\multicolumn{2}{c}{\small Use ctable or booktab for better looking tables.}
\end{tabular}
\end{table}
\subsection{Long Section Names: First subsection of first section of
introduction which is intentionally long to see what happens if it needs
to break a
line}\label{long-section-names-first-subsection-of-first-section-of-introduction-which-is-intentionally-long-to-see-what-happens-if-it-needs-to-break-a-line}

Another approach to fitting large tables is to shrink the fontsize. We
provide the \texttt{scriptsizetabluar} environment to help you with
this. Note that you can have a different caption in your List of Tables
than in the text, if you wish.
\begin{table}\centering
\caption[Alternative caption for List of Tables]{A table using scriptsizetabluar.}
\begin{scriptsizetabular}{lr}\hline\hline
Title & Author \\ \hline
TANSTAAFL & Milton Friedman \\
Oh Yes There Is & John Maynard Keynes \\ \hline
\end{scriptsizetabular}
\end{table}
\subsubsection{Subsubsection for test
purposes}\label{subsubsection-for-test-purposes}

Another paragraph. Another sentence. Another sentence. Another sentence.
Another sentence. Another sentence. Another sentence. Another sentence.
Another sentence. Another sentence. Another sentence. Another sentence.
Another sentence. Another sentence. Another sentence.

\chapter{PREVIOUS WORK}\label{previous-work}

Some other research was once performed.

\section{Section}\label{section}

Some was good and some was bad.
\begin{figure}
\centering MY FIRST FIGURE
\caption{Figure caption must be below figure.}
\end{figure}
\subsection{Subsection}\label{subsection}

Some was neither good or bad.
\begin{figure}
\centering MY SECOND FIGURE
\caption{Second figure.}
\end{figure}
\subsubsection{Subsubsection}\label{subsubsection}

Surely mine will be better. Here is a bulleted list of reasons why.
\begin{itemize}
\item Reason 1.
Some people hate the way test wraps in our bulleted lists,
but that is what the Guide says we have to do---apparently based on Turabian.
You can argue with CAS if you wish \dots
\item Reason 2.
\item Reason 3.
\item Reason 4.
\end{itemize}
\chapter{DISCUSSION OF FINDINGS}\label{discussion-of-findings}

\section{Section}\label{section-1}
\begin{figure}
\centering MY THIRD FIGURE
\caption{First figure in second section.}
\end{figure}
\subsection{Subsection}\label{subsection-1}
\begin{figure}
\centering MY FOURTH FIGURE
\caption{Second figure in second section.}
\end{figure}
\subsubsection{Subsubsection}\label{subsubsection-1}

\section{Section}\label{section-2}
\begin{figure}
\centering MY FIFTH FIGURE
\caption{First figure in third section.}
\end{figure}
\subsection{Subsection}\label{subsection-2}

\section{Section}\label{section-3}

\subsection{Subsection}\label{subsection-3}

\subsection{Subsection}\label{subsection-4}

\section{Section}\label{section-4}

\subsection{Subsection}\label{subsection-5}

\subsection{Subsection}\label{subsection-6}
\begin{figure}
\centering MY SIXTH FIGURE
\caption{Second figure in third section.}
\end{figure}
\subsubsection{Subsubsection}\label{subsubsection-2}

\chapter{CONCLUSION}\label{conclusion}

All is well that ends.

This will end soon. This will end soon. This will end soon. This will
end soon. This will end soon. This will end soon. This will end soon.
This will end soon. This will end soon. This will end soon. This will
end soon. This will end soon. This will end soon. This will end soon.
This will end soon. This will end soon. This will end soon. This will
end soon. This will end soon. This will end soon. This will end soon.
This will end soon. This will end soon. This will end soon. This will
end soon. This will end soon. This will end soon. This will end soon.
This will end soon. This will end soon. This will end soon. This will
end soon. This will end soon. This will end soon. This will end soon.
This will end soon. This will end soon. This will end soon. This will
end soon. This will end soon. This will end soon. This will end soon.
This will end soon. This will end soon. This will end soon. This will
end soon. This will end soon. This will end soon. This will end soon.
This will end soon.

This will end soon. This will end soon. This will end soon. This will
end soon. This will end soon. This will end soon. This will end soon.
This will end soon. This will end soon. This will end soon. This will
end soon. This will end soon. This will end soon. This will end soon.
This will end soon. This will end soon. This will end soon. This will
end soon. This will end soon. This will end soon. This will end soon.
This will end soon. This will end soon. This will end soon. This will
end soon. This will end soon. This will end soon. This will end soon.
This will end soon. This will end soon. This will end soon. This will
end soon. This will end soon. This will end soon. This will end soon.
This will end soon. This will end soon. This will end soon. This will
end soon. This will end soon. This will end soon. This will end soon.
This will end soon. This will end soon. This will end soon. This will
end soon. This will end soon. This will end soon. This will end soon.
This will end soon.

See?

\appendix

\chapter{SOME ANCILLARY STUFF}\label{some-ancillary-stuff}

\chapter{SOME MORE ANCILLARY STUFF}\label{some-more-ancillary-stuff}

\chapter*{REFERENCES}\label{references}
\addcontentsline{toc}{chapter}{REFERENCES}

\noindent

\ssp
% This loads all chapters up to and including the Conclusion, the content of which is created in the respective .Rmd files
% The appendix and references are loaded in separate .Rmd files
%%% SH end

\end{document}

